%---------- Inleiding ---------------------------------------------------------

\section{Introductie}%
\label{sec:introductie}
In een wereld die steeds digitaler wordt zijn planning applicaties een handige tool voor het inplannen van dagelijkse activiteiten en het bijhouden van afspraken, deadlines etc. Deze applicaties zijn echter niet altijd even toegankelijk voor alle gebruikers, onder andere voor mensen met ADD. Deze groep individuen ervaart vaak moeilijkheden bij het gebruik van deze applicaties want vaak zijn ze te overweldigend en ingewikkeld. Dit komt omdat ze ontworpen zijn voor de gemiddelde mens. In dit onderzoek tracht ik deze moeilijkheden te verlichten volgens Inclusive Design. Omdat mensen met ADD nood hebben aan een goede planning om taken gedaan te krijgen, is dit onderzoek van belang. \newline 

Het ontwerpen van planning applicaties specifiek gericht op de behoeften en mogelijkheden van mensen met ADD is echter een complexe taak. Het vereist een diepgaand begrip van de individuele uitdagingen waarmee zij worden geconfronteerd. Het identificeren en verlichten van deze uitdagingen vereist samenwerking op het gebied van design, ontwikkeling, psychologie en technologie. \newline 

Het doel van dit onderzoeksvoorstel is om een diepgaand begrip te verkrijgen van de behoeften, uitdagingen en mogelijkheden voor mensen met ADD bij het gebruik van planning applicaties. Hiermee wordt het mogelijk om specifieke ontwerpkeuzes te evalueren en optimaliseren voor deze doelgroep. \newline 

Het uiteindelijke resultaat van dit onderzoeksvoorstel is het ontwikkelen van concrete aanbevelingen en richtlijnen voor het ontwerpen van planning applicaties die optimaal aansluiten bij de behoeften van mensen met ADD. Deze aanbevelingen kunnen worden gebruikt door ontwerpers, ontwikkelaars en beleidsmakers om inclusieve technologische oplossingen te creëren die de participatie en zelfstandigheid van deze doelgroep bevordert.
Door dit onderzoek hopen we bij te dragen aan een meer inclusieve samenleving waarin iedereen gelijke kansen krijgt om actief deel te nemen en te profiteren van de mogelijkheden die technologie biedt.


%---------- Stand van zaken ---------------------------------------------------

\section{literatuurstudie}%
\label{sec:literatuurstudie}

\subsection{Inclusief design} % \textbf{Inclusief design } \newline
Inclusief design dateert van voor softwareontwikkeling. Om te begrijpen hoe we het kunnen toepassen op software, kan het nuttig zijn om het bij fysieke producten te onderzoeken \autocite{Clarkson2003}. Hoewel deze ontwerpkeuzes vaak niet rechtstreeks toepasselijk zijn op het ontwerp van applicaties, kunnen de denkwijzen en methodologieën nuttig zijn voor het begrijpen van de benodigdheden bij mensen met beperkingen \newline

\subsection{Beperking} %\textbf{Beperkingen } \newline
Om te weten waarmee mensen met ADD het moeilijk hebben, moeten we eerst de beperking zelf en hun gedrag los van de online wereld  begrijpen \autocite{VanHerwegen2019} . Hierbij is het belangrijk om eerst te begrijpen wat Attention Deficit Disorder precies inhoudt, waarna we verder de uitdagingen en benodigdheden van personen met ADD kunnen onderzoeken \autocite{diamond2005attention}.
\newline

\subsection{Beperking en technologie} %\textbf{Beperkingen en technologie} \newline
Om een concreter beeld te krijgen van welke uitdagingen mensen met ADD aangaan bij het gebruik van planning-applicaties, kunnen we ook kijken naar hoe ze omgaan met browser- \autocite{Harrysson2004} en mobiele \autocite{Rapp2019} applicaties. \newline

\subsection{Inclusief design toepassen} %\textbf{Inclusief design toepassen } \newline
Na het onderzoeken van de uitdagingen en benodigdheden die veroorzaakt worden door een concentratiestoornis, kunnen de richtlijnen van inclusief design binnen ICT \autocite{Gulliksen2004, Nicolle2001, Roessvoll2013} toegepast worden om de ervaring gebruiksvriendelijker te maken voor mensen met specifieke deze stoornis.


% Voor literatuurverwijzingen zijn er twee belangrijke commando's:
% \autocite{KEY} => (Auteur, jaartal) Gebruik dit als de naam van de auteur
%   geen onderdeel is van de zin.
% \textcite{KEY} => Auteur (jaartal)  Gebruik dit als de auteursnaam wel een
%   functie heeft in de zin (bv. ``Uit onderzoek door Doll & Hill (1954) bleek
%   ...'')

%---------- Methodologie ------------------------------------------------------
\section{Methodologie}%
\label{sec:methodologie}

De aanpak van dit onderzoek is opgesplitst in vijf fasen, namelijk: de literatuurstudie, een doelgroepanalyse, een ontwerpoptimalisatie, het ontwerp implementeren en data verzamelen en tot slot het trekken van conclusies en het formuleren van aanbevelingen. Hieronder volgt een toelichting van de besproken fases. \newline 


De eerste fase van het onderzoek omvat een uitgebreide literatuurstudie om een stevig theoretisch kader te ontwikkelen. Deze studie richt zich op relevante literatuur over de behoeften, uitdagingen en mogelijkheden van mensen met ADD bij het gebruik van planning applicaties. Daarnaast worden bestaande ontwerpprincipes, richtlijnen en methoden voor het ontwerpen van inclusieve applicaties geïdentificeerd en geanalyseerd. \newline 

In de tweede fase van het onderzoek wordt een grondige analyse van de doelgroep uitgevoerd. Dit omvat het verzamelen van kwantitatieve en kwalitatieve gegevens om inzicht te krijgen in onder andere de specifieke behoeften, vaardigheden en beperkingen van mensen met ADD. Verschillende onderzoeksmethoden kunnen worden toegepast, zoals enquêtes, interviews, observaties en gebruikerstests, maar ook academische onderzoeken. Deze gegevens zullen als basis voor het ontwerpproces dienen. \newline 

Op basis van de inzichten uit de literatuurstudie en de doelgroepanalyse begint de derde fase, waarin ontwerpoptimalisatie plaatsvindt. In deze fase worden ontwerpprincipes en richtlijnen geformuleerd die specifiek zijn afgestemd op de behoeften van de doelgroep. Dit omvat het identificeren van gebruiksvriendelijke en toegankelijke interface-elementen, het vereenvoudigen van complexe taken en het bevorderen van betrokkenheid en motivatie bij het gebruik van planning applicaties. \newline 

Na het definiëren van de ontwerpoptimalisaties gaat het onderzoek over naar de implementatiefase. Hier worden de ontwerpconcepten en verbeteringen geïmplementeerd in concrete planning applicaties. Deze applicaties worden vervolgens getest en geëvalueerd met de doelgroep. De verzamelde gegevens omvatten zowel kwantitatieve als kwalitatieve metingen, zoals gebruikerstevredenheid, gebruiksgemak, efficiëntie en effectiviteit van de applicaties. \newline 

In de laatste fase worden de verzamelde gegevens geanalyseerd en geïnterpreteerd. Op basis van deze analyse worden conclusies getrokken met betrekking tot de effectiviteit en bruikbaarheid van de ontwikkelde planning applicaties voor mensen met ADD. Daarnaast worden aanbevelingen geformuleerd voor verdere optimalisatie en verbetering van deze applicaties, evenals voor toekomstig onderzoek op dit gebied.


%---------- Verwachte resultaten ----------------------------------------------
\section{Verwacht resultaat, Conclusie}%
\label{sec:verwachte_resultaten_conclusie}

De verwachte resultaten omvatten het volgende: een heldere en intuïtieve gebruiksflow, waarbij er gebruik wordt gemaakt van duidelijke en herkenbare icoontjes om de navigatie te vergemakkelijken. Om de visuele belasting te verminderen, streven we naar schermen die niet te druk zijn. \newline
Er zal waarschijnlijk een extra focus liggen op personalisatie, waarbij gebruikers de mogelijkheid hebben om functies aan of uit te schakelen op basis van hun voorkeuren. Bijvoorbeeld, het activeren van de 'Todo'-optie terwijl notitie herinneringen uitgeschakeld kunnen worden. \newline

Bovendien zal er moeten gezorgd worden voor duidelijk verdeelde tabs, elk met specifieke taken en functies om een overzichtelijke structuur te bieden. \newline
 
Door deze aanpassingen wordt er gestreefd naar een geoptimaliseerde en op maat gemaakte gebruikerservaring die specifiek inspeelt op de behoeften van mensen met ADD. Hierdoor zal het gebruik van planning-apps effectiever en aangenamer worden en wil ik bijdragen aan de ontwikkeling van deze applicaties die optimaal aansluiten bij hun behoeften.

