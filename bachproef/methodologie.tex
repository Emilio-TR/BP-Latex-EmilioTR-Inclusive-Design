%%=============================================================================
%% Methodologie
%%=============================================================================

\chapter{\IfLanguageName{dutch}{Methodologie}{Methodology}}%
\label{ch:methodologie}

%% TODO: In dit hoofstuk geef je een korte toelichting over hoe je te werk bent
%% gegaan. Verdeel je onderzoek in grote fasen, en licht in elke fase toe wat
%% de doelstelling was, welke deliverables daar uit gekomen zijn, en welke
%% onderzoeksmethoden je daarbij toegepast hebt. Verantwoord waarom je
%% op deze manier te werk gegaan bent.
%% 
%% Voorbeelden van zulke fasen zijn: literatuurstudie, opstellen van een
%% requirements-analyse, opstellen long-list (bij vergelijkende studie),
%% selectie van geschikte tools (bij vergelijkende studie, "short-list"),
%% opzetten testopstelling/PoC, uitvoeren testen en verzamelen
%% van resultaten, analyse van resultaten, ...
%%
%% !!!!! LET OP !!!!!
%%
%% Het is uitdrukkelijk NIET de bedoeling dat je het grootste deel van de corpus
%% van je bachelorproef in dit hoofstuk verwerkt! Dit hoofdstuk is eerder een
%% kort overzicht van je plan van aanpak.
%%
%% Maak voor elke fase (behalve het literatuuronderzoek) een NIEUW HOOFDSTUK aan
%% en geef het een gepaste titel.

In deze sectie zal het volledige proces van het ontwikkelen van de kalender worden besproken. \newline

Om te beginnen zullen de eisen worden geprioriteerd met behulp van de MoSCoW-methode. Hierbij worden ze ingedeeld in vier categorieën: Must-have, Should-have, Could-have, en Won't-have. Daarna zal een applicatie ontwikkeld worden die grotendeels aan deze eisen voldoet, rekening houdend met de beperkingen in tijd en middelen. \newline

Voor het maken van de proof of concept zal een stappenplan worden opgebouwd voor het ontwikkelen van een eenvoudige kalenderapplicatie, inclusief de integratie van specifieke extra eisen. Ook een link naar de GitHub-repository van de uiteindelijke applicatie zal beschikbaar zijn. Zodat degenen die het onderzoek willen voortzetten of de applicatie verder willen ontwikkelen gemakkelijk een totale toegang hebben tot de ontwikkelde applicatie.

\section{MoSCoW-Methode}
Uit de lijst opgesteld in de literatuurstudie kan meteen worden afgeleid dat de belangrijkste functionaliteiten van de kalender alles is dat te maken heeft met het plannen. Deze zullen bij de must-have worden geplaatst omdat dit noodzakelijk is om een kalender te kunnen zijn. Bij de should-have zullen de zaken worden geplaatst die de Quality of Life (QoL) verhogen zoals de inclusieve oplossing die gevonden werden, namelijk de Brain Dump, overzicht van taken, etc. \newline

Bij de could-have van deze Proof of Concept zullen vooral zaken zitten die niet ontwikkeld gaan worden tijdens het onderzoek, maar er later zeker bij kunnen komen zoals een groepskalender die gesynchroniseerd is met die van andere personen of het importeren van externe kalenders.  De won't-have is nog niet echt besproken maar hier zullen zaken bij horen die absoluut vermeden moeten worden omdat ze voor te veel afleiding zorgen. \newline

Dan ziet de geprioriteerde lijst van eisen aan de hand van de MoSCoW-methode er als volgt uit:

\subsection{Must have}

    \begin{itemize}
        \item 	Weekoverzicht
        \item 	Maandoverzicht
        \item 	Planning raadplegen
        \item 	Taken toevoegen
        \item 	Afspraken en andere zaken kunnen inplannen
        \item 	Ingeplande taken, afspraken en andere zaken kunnen raadplegen
        \item 	Taken, afspraken en ander ingeplande zaken kunnen beheren (aanpassen, verwijderen, …)
    \end{itemize}


\subsection{Should have}

    \begin{itemize}
        \item	‘Andere zaken’, events, taken en afspraken kunnen opdelen in categorieën
        \item	Navigeren tussen weken, maanden, jaren (afhankelijk van view)
        \item	Overzicht van alle in te plannen taken/ events
        \item	Toevoegen van een ‘Brain Dump’ 
        \item	Prioritering van taken
        \item	Een badge verdienen bij het verrichten van taken
        \item	Mogelijkheid om grote/ lange taken op te delen in kleinere taken
        \item	Categorieën een kleur geven
        \item	Notificaties 
    \end{itemize}
    

\subsection{Could have}

    \begin{itemize}
        \item 	Groepskalender
        \item 	Jaaroverzicht
        \item 	Mogelijkheden om taken, afspraken en andere zaken te herhalen voor x aantal dagen/ weken/ maanden
        \item 	Externe kalenders importeren
        \item 	Light/Dark mode 
    \end{itemize}
    
    
\subsection{Won't have}

    \begin{itemize}
        \item	Advertenties
        \item	Pop-ups
        \item	Links naar externe sociale media
         
    \end{itemize}


Nu de vereisten duidelijk zijn gespecificeerd, is het tijd om over te gaan naar de ontwikkeling van de Proof of Concept. Het volgende hoofdstuk zal een selectie presenteren van de requirements die in de voorgaande sectie zijn geordend. \newline

De opbouw van de proof of concept zal stap voor stap worden overlopen, te beginnen met de voorbereiding van de gebruikersinterface (UI) en lay-out.  \newline
Vervolgens zullen we de ontwikkeling behandelen, waarbij we de benodigde vereisten voor de opbouw van een simpele applicatie implementeren en functionaliteiten toevoegen om aan de gestelde eisen te voldoen. \newline

Nadat de Proof of Concept is opgebouwd, zal deze online worden gezet en getest worden door de doelgroep aan de hand van een evaluatieformulier. \newline
Daarna zullen we alle verkregen resultaten analyseren en de nodige conclusies trekken met betrekking tot de ontworpen features.

