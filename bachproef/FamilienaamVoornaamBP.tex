%===============================================================================
% LaTeX sjabloon voor de bachelorproef toegepaste informatica aan HOGENT
% Meer info op https://github.com/HoGentTIN/latex-hogent-report
%===============================================================================

\documentclass[dutch,dit,thesis]{hogentreport}

% TODO:
% - If necessary, replace the option `dit`' with your own department!
%   Valid entries are dbo, dbt, dgz, dit, dlo, dog, dsa, soa
% - If you write your thesis in English (remark: only possible after getting
%   explicit approval!), remove the option "dutch," or replace with "english".

\usepackage{lipsum} % For blind text, can be removed after adding actual content

%% Pictures to include in the text can be put in the graphics/ folder
\graphicspath{{graphics/}}

%% For source code highlighting, requires pygments to be installed
%% Compile with the -shell-escape flag!
\usepackage[section]{minted}
%% If you compile with the make_thesis.{bat,sh} script, use the following
%% import instead:
%% \usepackage[section,outputdir=../output]{minted}
\usemintedstyle{solarized-light}
\definecolor{bg}{RGB}{253,246,227} %% Set the background color of the codeframe

%% Change this line to edit the line numbering style:
\renewcommand{\theFancyVerbLine}{\ttfamily\scriptsize\arabic{FancyVerbLine}}

%% Macro definition to load external java source files with \javacode{filename}:
\newmintedfile[javacode]{java}{
    bgcolor=bg,
    fontfamily=tt,
    linenos=true,
    numberblanklines=true,
    numbersep=5pt,
    gobble=0,
    framesep=2mm,
    funcnamehighlighting=true,
    tabsize=4,
    obeytabs=false,
    breaklines=true,
    mathescape=false
    samepage=false,
    showspaces=false,
    showtabs =false,
    texcl=false,
}

% Other packages not already included can be imported here

%%---------- Document metadata -------------------------------------------------
% TODO: Replace this with your own information
\author{Emilio Tena Romero}
\supervisor{Dhr. S. Verschraege}
\cosupervisor{Dhr. G. Geens}
\title %[Optionele ondertitel]%
    {Inclusief Design: Optimalisatie van een kalenderapplicatie voor personen met Attention Deficit (Hyperactivity) Disorder.}
\academicyear{\advance\year by -1 \the\year--\advance\year by 1 \the\year}
\examperiod{1}
\degreesought{\IfLanguageName{dutch}{Professionele bachelor in de toegepaste informatica}{Bachelor of applied computer science}}
\partialthesis{false} %% To display 'in partial fulfilment'
%\institution{Internshipcompany BVBA.}

%% Add global exceptions to the hyphenation here
\hyphenation{back-slash}

%% The bibliography (style and settings are  found in hogentthesis.cls)
\addbibresource{bachproef.bib}            %% Bibliography file
\addbibresource{../voorstel/voorstel.bib} %% Bibliography research proposal
\defbibheading{bibempty}{}

%% Prevent empty pages for right-handed chapter starts in twoside mode
\renewcommand{\cleardoublepage}{\clearpage}

\renewcommand{\arraystretch}{1.2}

%% Content starts here.
\begin{document}

%---------- Front matter -------------------------------------------------------

\frontmatter

\hypersetup{pageanchor=false} %% Disable page numbering references
%% Render a Dutch outer title page if the main language is English
\IfLanguageName{english}{%
    %% If necessary, information can be changed here
    \degreesought{Professionele Bachelor toegepaste informatica}%
    \begin{otherlanguage}{dutch}%
       \maketitle%
    \end{otherlanguage}%
}{}

%% Generates title page content
\maketitle
\hypersetup{pageanchor=true}

%%=============================================================================
%% Voorwoord
%%=============================================================================

\chapter*{\IfLanguageName{dutch}{Woord vooraf}{Preface}}%
\label{ch:voorwoord}

%% TODO:
%% Het voorwoord is het enige deel van de bachelorproef waar je vanuit je
%% eigen standpunt (``ik-vorm'') mag schrijven. Je kan hier bv. motiveren
%% waarom jij het onderwerp wil bespreken.
%% Vergeet ook niet te bedanken wie je geholpen/gesteund/... heeft

\lipsum[1-2]
%%=============================================================================
%% Samenvatting
%%=============================================================================

% TODO: De "abstract" of samenvatting is een kernachtige (~ 1 blz. voor een
% thesis) synthese van het document.
%
% Een goede abstract biedt een kernachtig antwoord op volgende vragen:
%
% 1. Waarover gaat de bachelorproef?
% 2. Waarom heb je er over geschreven?
% 3. Hoe heb je het onderzoek uitgevoerd?
% 4. Wat waren de resultaten? Wat blijkt uit je onderzoek?
% 5. Wat betekenen je resultaten? Wat is de relevantie voor het werkveld?
%
% Daarom bestaat een abstract uit volgende componenten:
%
% - inleiding + kaderen thema
% - probleemstelling
% - (centrale) onderzoeksvraag
% - onderzoeksdoelstelling
% - methodologie
% - resultaten (beperk tot de belangrijkste, relevant voor de onderzoeksvraag)
% - conclusies, aanbevelingen, beperkingen
%
% LET OP! Een samenvatting is GEEN voorwoord!

%%---------- Nederlandse samenvatting -----------------------------------------
%
% TODO: Als je je bachelorproef in het Engels schrijft, moet je eerst een
% Nederlandse samenvatting invoegen. Haal daarvoor onderstaande code uit
% commentaar.
% Wie zijn bachelorproef in het Nederlands schrijft, kan dit negeren, de inhoud
% wordt niet in het document ingevoegd.

\IfLanguageName{english}{%
\selectlanguage{dutch}
\chapter*{Samenvatting}
\lipsum[1-4]
\selectlanguage{english}
}{}

%%---------- Samenvatting -----------------------------------------------------
% De samenvatting in de hoofdtaal van het document

\chapter*{\IfLanguageName{dutch}{Samenvatting}{Abstract}}

\lipsum[1-4]


%---------- Inhoud, lijst figuren, ... -----------------------------------------

\tableofcontents

% In a list of figures, the complete caption will be included. To prevent this,
% ALWAYS add a short description in the caption!
%
%  \caption[short description]{elaborate description}
%
% If you do, only the short description will be used in the list of figures

\listoffigures

% If you included tables and/or source code listings, uncomment the appropriate
% lines.
%\listoftables
%\listoflistings

% Als je een lijst van afkortingen of termen wil toevoegen, dan hoort die
% hier thuis. Gebruik bijvoorbeeld de ``glossaries'' package.
% https://www.overleaf.com/learn/latex/Glossaries

%---------- Kern ---------------------------------------------------------------

\mainmatter{}

% De eerste hoofdstukken van een bachelorproef zijn meestal een inleiding op
% het onderwerp, literatuurstudie en verantwoording methodologie.
% Aarzel niet om een meer beschrijvende titel aan deze hoofdstukken te geven of
% om bijvoorbeeld de inleiding en/of stand van zaken over meerdere hoofdstukken
% te verspreiden!

%%=============================================================================
%% Inleiding
%%=============================================================================

\chapter{\IfLanguageName{dutch}{Inleiding}{Introduction}}%
\label{ch:inleiding}

De inleiding moet de lezer net genoeg informatie verschaffen om het onderwerp te begrijpen en in te zien waarom de onderzoeksvraag de moeite waard is om te onderzoeken. In de inleiding ga je literatuurverwijzingen beperken, zodat de tekst vlot leesbaar blijft. Je kan de inleiding verder onderverdelen in secties als dit de tekst verduidelijkt. Zaken die aan bod kunnen komen in de inleiding~\autocite{Pollefliet2011}:

\begin{itemize}
  \item context, achtergrond
  \item afbakenen van het onderwerp
  \item verantwoording van het onderwerp, methodologie
  \item probleemstelling
  \item onderzoeksdoelstelling
  \item onderzoeksvraag
  \item \ldots
\end{itemize}

\section{\IfLanguageName{dutch}{Probleemstelling}{Problem Statement}}%
\label{sec:probleemstelling}

Uit je probleemstelling moet duidelijk zijn dat je onderzoek een meerwaarde heeft voor een concrete doelgroep. De doelgroep moet goed gedefinieerd en afgelijnd zijn. Doelgroepen als ``bedrijven,'' ``KMO's'', systeembeheerders, enz.~zijn nog te vaag. Als je een lijstje kan maken van de personen/organisaties die een meerwaarde zullen vinden in deze bachelorproef (dit is eigenlijk je steekproefkader), dan is dat een indicatie dat de doelgroep goed gedefinieerd is. Dit kan een enkel bedrijf zijn of zelfs één persoon (je co-promotor/opdrachtgever).

\section{\IfLanguageName{dutch}{Onderzoeksvraag}{Research question}}%
\label{sec:onderzoeksvraag}

Wees zo concreet mogelijk bij het formuleren van je onderzoeksvraag. Een onderzoeksvraag is trouwens iets waar nog niemand op dit moment een antwoord heeft (voor zover je kan nagaan). Het opzoeken van bestaande informatie (bv. ``welke tools bestaan er voor deze toepassing?'') is dus geen onderzoeksvraag. Je kan de onderzoeksvraag verder specifiëren in deelvragen. Bv.~als je onderzoek gaat over performantiemetingen, dan 

\section{\IfLanguageName{dutch}{Onderzoeksdoelstelling}{Research objective}}%
\label{sec:onderzoeksdoelstelling}

Wat is het beoogde resultaat van je bachelorproef? Wat zijn de criteria voor succes? Beschrijf die zo concreet mogelijk. Gaat het bv.\ om een proof-of-concept, een prototype, een verslag met aanbevelingen, een vergelijkende studie, enz.

\section{\IfLanguageName{dutch}{Opzet van deze bachelorproef}{Structure of this bachelor thesis}}%
\label{sec:opzet-bachelorproef}

% Het is gebruikelijk aan het einde van de inleiding een overzicht te
% geven van de opbouw van de rest van de tekst. Deze sectie bevat al een aanzet
% die je kan aanvullen/aanpassen in functie van je eigen tekst.

De rest van deze bachelorproef is als volgt opgebouwd:

In Hoofdstuk~\ref{ch:stand-van-zaken} wordt een overzicht gegeven van de stand van zaken binnen het onderzoeksdomein, op basis van een literatuurstudie.

In Hoofdstuk~\ref{ch:methodologie} wordt de methodologie toegelicht en worden de gebruikte onderzoekstechnieken besproken om een antwoord te kunnen formuleren op de onderzoeksvragen.

% TODO: Vul hier aan voor je eigen hoofstukken, één of twee zinnen per hoofdstuk

In Hoofdstuk~\ref{ch:conclusie}, tenslotte, wordt de conclusie gegeven en een antwoord geformuleerd op de onderzoeksvragen. Daarbij wordt ook een aanzet gegeven voor toekomstig onderzoek binnen dit domein.
\chapter{\IfLanguageName{dutch}{Stand van zaken}{State of the art}}%
\label{ch:stand-van-zaken}

% Tip: Begin elk hoofdstuk met een paragraaf inleiding die beschrijft hoe
% dit hoofdstuk past binnen het geheel van de bachelorproef. Geef in het
% bijzonder aan wat de link is met het vorige en volgende hoofdstuk.

% Pas na deze inleidende paragraaf komt de eerste sectiehoofding.

Dit hoofdstuk bevat je literatuurstudie. De inhoud gaat verder op de inleiding, maar zal het onderwerp van de bachelorproef *diepgaand* uitspitten. De bedoeling is dat de lezer na lezing van dit hoofdstuk helemaal op de hoogte is van de huidige stand van zaken (state-of-the-art) in het onderzoeksdomein. Iemand die niet vertrouwd is met het onderwerp, weet nu voldoende om de rest van het verhaal te kunnen volgen, zonder dat die er nog andere informatie moet over opzoeken \autocite{Pollefliet2011}.

Je verwijst bij elke bewering die je doet, vakterm die je introduceert, enz.\ naar je bronnen. In \LaTeX{} kan dat met het commando \texttt{$\backslash${textcite\{\}}} of \texttt{$\backslash${autocite\{\}}}. Als argument van het commando geef je de ``sleutel'' van een ``record'' in een bibliografische databank in het Bib\LaTeX{}-formaat (een tekstbestand). Als je expliciet naar de auteur verwijst in de zin (narratieve referentie), gebruik je \texttt{$\backslash${}textcite\{\}}. Soms is de auteursnaam niet expliciet een onderdeel van de zin, dan gebruik je \texttt{$\backslash${}autocite\{\}} (referentie tussen haakjes). Dit gebruik je bv.~bij een citaat, of om in het bijschrift van een overgenomen afbeelding, broncode, tabel, enz. te verwijzen naar de bron. In de volgende paragraaf een voorbeeld van elk.

\textcite{Knuth1998} schreef een van de standaardwerken over sorteer- en zoekalgoritmen. Experten zijn het erover eens dat cloud computing een interessante opportuniteit vormen, zowel voor gebruikers als voor dienstverleners op vlak van informatietechnologie~\autocite{Creeger2009}.

Let er ook op: het \texttt{cite}-commando voor de punt, dus binnen de zin. Je verwijst meteen naar een bron in de eerste zin die erop gebaseerd is, dus niet pas op het einde van een paragraaf.

\lipsum[7-20]

%%=============================================================================
%% Methodologie
%%=============================================================================

\chapter{\IfLanguageName{dutch}{Methodologie}{Methodology}}%
\label{ch:methodologie}

%% TODO: In dit hoofstuk geef je een korte toelichting over hoe je te werk bent
%% gegaan. Verdeel je onderzoek in grote fasen, en licht in elke fase toe wat
%% de doelstelling was, welke deliverables daar uit gekomen zijn, en welke
%% onderzoeksmethoden je daarbij toegepast hebt. Verantwoord waarom je
%% op deze manier te werk gegaan bent.
%% 
%% Voorbeelden van zulke fasen zijn: literatuurstudie, opstellen van een
%% requirements-analyse, opstellen long-list (bij vergelijkende studie),
%% selectie van geschikte tools (bij vergelijkende studie, "short-list"),
%% opzetten testopstelling/PoC, uitvoeren testen en verzamelen
%% van resultaten, analyse van resultaten, ...
%%
%% !!!!! LET OP !!!!!
%%
%% Het is uitdrukkelijk NIET de bedoeling dat je het grootste deel van de corpus
%% van je bachelorproef in dit hoofstuk verwerkt! Dit hoofdstuk is eerder een
%% kort overzicht van je plan van aanpak.
%%
%% Maak voor elke fase (behalve het literatuuronderzoek) een NIEUW HOOFDSTUK aan
%% en geef het een gepaste titel.

In deze sectie zal het volledige proces van het ontwikkelen van de kalender worden besproken. \newline

Om te beginnen zullen de eisen worden geprioriteerd met behulp van de MoSCoW-methode. Hierbij worden ze ingedeeld in vier categorieën: Must-have, Should-have, Could-have, en Won't-have. Daarna zal een applicatie ontwikkeld worden die grotendeels aan deze eisen voldoet, rekening houdend met de beperkingen in tijd en middelen. \newline

Voor het maken van de proof of concept zal een stappenplan worden opgebouwd voor het ontwikkelen van een eenvoudige kalenderapplicatie, inclusief de integratie van specifieke extra eisen. Ook een link naar de GitHub-repository van de uiteindelijke applicatie zal beschikbaar zijn. Zodat degenen die het onderzoek willen voortzetten of de applicatie verder willen ontwikkelen gemakkelijk een totale toegang hebben tot de ontwikkelde applicatie.

\section{MoSCoW-Methode}
Uit de lijst opgesteld in de literatuurstudie kan meteen worden afgeleid dat de belangrijkste functionaliteiten van de kalender alles is dat te maken heeft met het plannen. Deze zullen bij de must-have worden geplaatst omdat dit noodzakelijk is om een kalender te kunnen zijn. Bij de should-have zullen de zaken worden geplaatst die de Quality of Life (QoL) verhogen zoals de inclusieve oplossing die gevonden werden, namelijk de Brain Dump, overzicht van taken, etc. \newline

Bij de could-have van deze Proof of Concept zullen vooral zaken zitten die niet ontwikkeld gaan worden tijdens het onderzoek, maar er later zeker bij kunnen komen zoals een groepskalender die gesynchroniseerd is met die van andere personen of het importeren van externe kalenders.  De won't-have is nog niet echt besproken maar hier zullen zaken bij horen die absoluut vermeden moeten worden omdat ze voor te veel afleiding zorgen. \newline

Dan ziet de geprioriteerde lijst van eisen aan de hand van de MoSCoW-methode er als volgt uit:

\subsection{Must have}

    \begin{itemize}
        \item 	Weekoverzicht
        \item 	Maandoverzicht
        \item 	Planning raadplegen
        \item 	Taken toevoegen
        \item 	Afspraken en andere zaken kunnen inplannen
        \item 	Ingeplande taken, afspraken en andere zaken kunnen raadplegen
        \item 	Taken, afspraken en ander ingeplande zaken kunnen beheren (aanpassen, verwijderen, …)
    \end{itemize}


\subsection{Should have}

    \begin{itemize}
        \item	‘Andere zaken’, events, taken en afspraken kunnen opdelen in categorieën
        \item	Navigeren tussen weken, maanden, jaren (afhankelijk van view)
        \item	Overzicht van alle in te plannen taken/ events
        \item	Toevoegen van een ‘Brain Dump’ 
        \item	Prioritering van taken
        \item	Een badge verdienen bij het verrichten van taken
        \item	Mogelijkheid om grote/ lange taken op te delen in kleinere taken
        \item	Categorieën een kleur geven
        \item	Notificaties 
    \end{itemize}
    

\subsection{Could have}

    \begin{itemize}
        \item 	Groepskalender
        \item 	Jaaroverzicht
        \item 	Mogelijkheden om taken, afspraken en andere zaken te herhalen voor x aantal dagen/ weken/ maanden
        \item 	Externe kalenders importeren
        \item 	Light/Dark mode 
    \end{itemize}
    
    
\subsection{Won't have}

    \begin{itemize}
        \item	Advertenties
        \item	Pop-ups
        \item	Links naar externe sociale media
         
    \end{itemize}


Nu de vereisten duidelijk zijn gespecificeerd, is het tijd om over te gaan naar de ontwikkeling van de Proof of Concept. Het volgende hoofdstuk zal een selectie presenteren van de requirements die in de voorgaande sectie zijn geordend. \newline

De opbouw van de proof of concept zal stap voor stap worden overlopen, te beginnen met de voorbereiding van de gebruikersinterface (UI) en lay-out.  \newline
Vervolgens zullen we de ontwikkeling behandelen, waarbij we de benodigde vereisten voor de opbouw van een simpele applicatie implementeren en functionaliteiten toevoegen om aan de gestelde eisen te voldoen. \newline

Nadat de Proof of Concept is opgebouwd, zal deze online worden gezet en getest worden door de doelgroep aan de hand van een evaluatieformulier. \newline
Daarna zullen we alle verkregen resultaten analyseren en de nodige conclusies trekken met betrekking tot de ontworpen features.



% Voeg hier je eigen hoofdstukken toe die de ``corpus'' van je bachelorproef
% vormen. De structuur en titels hangen af van je eigen onderzoek. Je kan bv.
% elke fase in je onderzoek in een apart hoofdstuk bespreken.

%\input{...}
%\input{...}
%...

%%=============================================================================
%% Conclusie
%%=============================================================================

\chapter{Conclusie}%
\label{ch:conclusie}

% TODO: Trek een duidelijke conclusie, in de vorm van een antwoord op de
% onderzoeksvra(a)g(en). Wat was jouw bijdrage aan het onderzoeksdomein en
% hoe biedt dit meerwaarde aan het vakgebied/doelgroep? 
% Reflecteer kritisch over het resultaat. In Engelse teksten wordt deze sectie
% ``Discussion'' genoemd. Had je deze uitkomst verwacht? Zijn er zaken die nog
% niet duidelijk zijn?
% Heeft het onderzoek geleid tot nieuwe vragen die uitnodigen tot verder 
%onderzoek?

\lipsum[76-80]



%---------- Bijlagen -----------------------------------------------------------

\appendix

\chapter{Onderzoeksvoorstel}

Het onderwerp van deze bachelorproef is gebaseerd op een onderzoeksvoorstel dat vooraf werd beoordeeld door de promotor. Dat voorstel is opgenomen in deze bijlage.

%% TODO: 
\section*{Samenvatting}
Dit onderzoek richt zich op het gebruik van inclusief ontwerp om een kalender applicatie te optimaliseren voor
mensen met AD(H)D. Deze soort applicaties zijn steeds meer noodzakelijk voor het verrichten en inplannen van
dagelijkse activiteiten door de hedendaagse drukte. Voor mensen met AD(H)D kan het gebruik van deze applicaties echter problematisch zijn vanwege hun complexiteit. Ze zijn vaak te overweldigend voor mensen met
AD(H)D omdat het ontwerp van de meeste applicaties niet gericht is op hun specifieke behoeften.
Het onderzoek begint met het identificeren van deze behoeften. Vervolgens worden er richtlijnen en beste praktijken opgesteld die aan deze behoeften voldoen voor het inclusief ontwerp. Eens deze richtlijnen zijn opgesteld
kan men de ontwerpen hier op afstemmen en zal er een applicatie ontwikkeld worden die later getest zal worden.
De resultaten van dit onderzoek zullen inzicht geven in de benodigdheden binnen de digitale wereld voor mensen met AD(H)D en zullen ontwerpers in staat stellen om een inclusief ontwerp te maken dat voor iedereen
toegankelijk is. Uit dit onderzoek zal men waarschijnlijk kunnen concluderen dat het gebruik van onder andere
eenvoudige en logische ordening, duidelijke sjablonen en visuele elementen dit doelpubliek al sterk kan helpen
bij het gebruik van deze applicaties.


% Kopieer en plak hier de samenvatting (abstract) van je onderzoeksvoorstel.

% Verwijzing naar het bestand met de inhoud van het onderzoeksvoorstel
%---------- Inleiding ---------------------------------------------------------

\section{Introductie}%
\label{sec:introductie}
In een wereld die steeds digitaler wordt zijn planning applicaties een handige tool voor het inplannen van dagelijkse activiteiten en het bijhouden van afspraken, deadlines etc. Deze applicaties zijn echter niet altijd even toegankelijk voor alle gebruikers, onder andere voor mensen met ADD. Deze groep individuen ervaart vaak moeilijkheden bij het gebruik van deze applicaties want vaak zijn ze te overweldigend en ingewikkeld. Dit komt omdat ze ontworpen zijn voor de gemiddelde mens. In dit onderzoek tracht ik deze moeilijkheden te verlichten volgens Inclusive Design. Omdat mensen met ADD nood hebben aan een goede planning om taken gedaan te krijgen, is dit onderzoek van belang. \newline \newline
Het ontwerpen van planning applicaties specifiek gericht op de behoeften en mogelijkheden van mensen met ADD is echter een complexe taak. Het vereist een diepgaand begrip van de individuele uitdagingen waarmee zij worden geconfronteerd. Het identificeren en verlichten van deze uitdagingen vereist samenwerking op het gebied van design, ontwikkeling, psychologie en technologie. \newline \newline
Het doel van dit onderzoeksvoorstel is om een diepgaand begrip te verkrijgen van de behoeften, uitdagingen en mogelijkheden voor mensen met ADD bij het gebruik van planning applicaties. Hiermee wordt het mogelijk om specifieke ontwerpkeuzes te evalueren en optimaliseren voor deze doelgroep. \newline \newline
Het uiteindelijke resultaat van dit onderzoeksvoorstel is het ontwikkelen van concrete aanbevelingen en richtlijnen voor het ontwerpen van planning applicaties die optimaal aansluiten bij de behoeften van mensen met ADD. Deze aanbevelingen kunnen worden gebruikt door ontwerpers, ontwikkelaars en beleidsmakers om inclusieve technologische oplossingen te creëren die de participatie en zelfstandigheid van deze doelgroep bevordert.
Door dit onderzoek hopen we bij te dragen aan een meer inclusieve samenleving waarin iedereen gelijke kansen krijgt om actief deel te nemen en te profiteren van de mogelijkheden die technologie biedt.


%---------- Stand van zaken ---------------------------------------------------

\section{literatuurstudie}%
\label{sec:literatuurstudie}

\subsection{Inclusief design} % \textbf{Inclusief design } \newline
Inclusief design dateert van voor softwareontwikkeling. Om te begrijpen hoe we het kunnen toepassen op software, kan het nuttig zijn om het bij fysieke producten te onderzoeken \autocite{Clarkson2003}. Hoewel deze ontwerpkeuzes vaak niet rechtstreeks toepasselijk zijn op het ontwerp van applicaties, kunnen de denkwijzen en methodologieën nuttig zijn voor het begrijpen van de benodigdheden bij mensen met beperkingen \newline

\subsection{Beperking} %\textbf{Beperkingen } \newline
Om te weten waarmee mensen met ADD het moeilijk hebben, moeten we eerst de beperking zelf en hun gedrag los van de online wereld  begrijpen \autocite{VanHerwegen2019} . Hierbij is het belangrijk om eerst te begrijpen wat Attention Deficit Disorder precies inhoudt, waarna we verder de uitdagingen en benodigdheden van personen met ADD kunnen onderzoeken \autocite{diamond2005attention}.
\newline

\subsection{Beperking en technologie} %\textbf{Beperkingen en technologie} \newline
Om een concreter beeld te krijgen van welke uitdagingen mensen met ADD aangaan bij het gebruik van planning-applicaties, kunnen we ook kijken naar hoe ze omgaan met browser- \autocite{Harrysson2004} en mobiele \autocite{Rapp2019} applicaties. \newline

\subsection{Inclusief design toepassen} %\textbf{Inclusief design toepassen } \newline
Na het onderzoeken van de uitdagingen en benodigdheden die veroorzaakt worden door een concentratiestoornis, kunnen de richtlijnen van inclusief design binnen ICT \autocite{Gulliksen2004, Nicolle2001, Roessvoll2013} toegepast worden om de ervaring gebruiksvriendelijker te maken voor mensen met specifieke deze stoornis.


% Voor literatuurverwijzingen zijn er twee belangrijke commando's:
% \autocite{KEY} => (Auteur, jaartal) Gebruik dit als de naam van de auteur
%   geen onderdeel is van de zin.
% \textcite{KEY} => Auteur (jaartal)  Gebruik dit als de auteursnaam wel een
%   functie heeft in de zin (bv. ``Uit onderzoek door Doll & Hill (1954) bleek
%   ...'')

%---------- Methodologie ------------------------------------------------------
\section{Methodologie}%
\label{sec:methodologie}

De aanpak van dit onderzoek is opgesplitst in vijf fasen, namelijk: de literatuurstudie, een doelgroepanalyse, een ontwerpoptimalisatie, het ontwerp implementeren en data verzamelen en tot slot het trekken van conclusies en het formuleren van aanbevelingen. Hieronder volgt een toelichting van de besproken fases. \newline 


De eerste fase van het onderzoek omvat een uitgebreide literatuurstudie om een stevig theoretisch kader te ontwikkelen. Deze studie richt zich op relevante literatuur over de behoeften, uitdagingen en mogelijkheden van mensen met ADD bij het gebruik van planning applicaties. Daarnaast worden bestaande ontwerpprincipes, richtlijnen en methoden voor het ontwerpen van inclusieve applicaties geïdentificeerd en geanalyseerd. \newline 

In de tweede fase van het onderzoek wordt een grondige analyse van de doelgroep uitgevoerd. Dit omvat het verzamelen van kwantitatieve en kwalitatieve gegevens om inzicht te krijgen in onder andere de specifieke behoeften, vaardigheden en beperkingen van mensen met ADD. Verschillende onderzoeksmethoden kunnen worden toegepast, zoals enquêtes, interviews, observaties en gebruikerstests, maar ook academische onderzoeken. Deze gegevens zullen als basis voor het ontwerpproces dienen. \newline 

Op basis van de inzichten uit de literatuurstudie en de doelgroepanalyse begint de derde fase, waarin ontwerpoptimalisatie plaatsvindt. In deze fase worden ontwerpprincipes en richtlijnen geformuleerd die specifiek zijn afgestemd op de behoeften van de doelgroep. Dit omvat het identificeren van gebruiksvriendelijke en toegankelijke interface-elementen, het vereenvoudigen van complexe taken en het bevorderen van betrokkenheid en motivatie bij het gebruik van planning applicaties. \newline 

Na het definiëren van de ontwerpoptimalisaties gaat het onderzoek over naar de implementatiefase. Hier worden de ontwerpconcepten en verbeteringen geïmplementeerd in concrete planning applicaties. Deze applicaties worden vervolgens getest en geëvalueerd met de doelgroep. De verzamelde gegevens omvatten zowel kwantitatieve als kwalitatieve metingen, zoals gebruikerstevredenheid, gebruiksgemak, efficiëntie en effectiviteit van de applicaties. \newline 

In de laatste fase worden de verzamelde gegevens geanalyseerd en geïnterpreteerd. Op basis van deze analyse worden conclusies getrokken met betrekking tot de effectiviteit en bruikbaarheid van de ontwikkelde planning applicaties voor mensen met ADD. Daarnaast worden aanbevelingen geformuleerd voor verdere optimalisatie en verbetering van deze applicaties, evenals voor toekomstig onderzoek op dit gebied.


%---------- Verwachte resultaten ----------------------------------------------
\section{Verwacht resultaat, Conclusie}%
\label{sec:verwachte_resultaten_conclusie}

De verwachte resultaten omvatten het volgende: een heldere en intuïtieve gebruiksflow, waarbij er gebruik wordt gemaakt van duidelijke en herkenbare icoontjes om de navigatie te vergemakkelijken. Om de visuele belasting te verminderen, streven we naar schermen die niet te druk zijn. \newline
Er zal waarschijnlijk een extra focus liggen op personalisatie, waarbij gebruikers de mogelijkheid hebben om functies aan of uit te schakelen op basis van hun voorkeuren. Bijvoorbeeld, het activeren van de 'Todo'-optie terwijl notitie herinneringen uitgeschakeld kunnen worden. \newline
Bovendien zal er moeten gezorgd worden voor duidelijk verdeelde tabs, elk met specifieke taken en functies om een overzichtelijke structuur te bieden. \newline
 
Door deze aanpassingen wordt er gestreefd naar een geoptimaliseerde en op maat gemaakte gebruikerservaring die specifiek inspeelt op de behoeften van mensen met ADD. Hierdoor zal het gebruik van planning-apps effectiever en aangenamer worden en wil ik bijdragen aan de ontwikkeling van deze applicaties die optimaal aansluiten bij hun behoeften.



%%---------- Andere bijlagen --------------------------------------------------
% TODO: Voeg hier eventuele andere bijlagen toe. Bv. als je deze BP voor de
% tweede keer indient, een overzicht van de verbeteringen t.o.v. het origineel.
%\input{...}

%%---------- Backmatter, referentielijst ---------------------------------------

\backmatter{}

\setlength\bibitemsep{2pt} %% Add Some space between the bibliograpy entries
\printbibliography[heading=bibintoc]

\end{document}
